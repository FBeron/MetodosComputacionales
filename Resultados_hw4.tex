\documentclass[11pt, spanish]{article}
\usepackage[spanish]{babel}
\selectlanguage{spanish}
\usepackage{graphicx}
\graphicspath{ {images/} }

\title{Ecuacion de calor en 2D}
\author{Felipe Berón}
\date{\today}
 
\begin{document}

\maketitle

A continuacion se muestran los resultados para el taller. Este consistía en resolver la ecuacion de calor para una placa vidimensional con para tres casos y dos condiciones de frontera.

En ambos casos toda la placa comienza con una temperatura de 50°C en todas partes excepto en un peque rectángulo que inicia en 100°C.

Para todos los casos y condiciones se muestra la temperatura en todos los puntos de la placa para el t = 0s, t = 100s y t= 2500s. Adicionalmente se muetra la temperatura promedi de toda la placa para cada instante de tiempo.

\section{Sin fuente de temperatura}
En este caso no hay nada que mantenga el rectángulo que inica en 100°C a esta temperatura y por lo tanto la temperatura de estos puntos disminulle mientras que los de los demás aumenta.
\subsection{Frontera fija}
La frontera de la placa se mantiene a 50°C independientemente de la temperatura de los punntos de la placa adyacentes.

\includegraphics{TPlacaSinFuenteFijas0.png}
\includegraphics{TPlacaSinFuenteFijas100.png}
\includegraphics{TPlacaSinFuenteFijas2500.png}
\includegraphics{TpromedioSinFuenteFijas.png}

\subsection{Frontera libre}
La frontera de la placa tiene la misma temeratura que los nodos adyacentes a esta, por lo tanto la transferencia de calor en esta dirección es cero.

\includegraphics{TPlacaSinFuenteLibre0.png}
\includegraphics{TPlacaSinFuenteLibre100.png}
\includegraphics{TPlacaSinFuenteLibre2500.png}
\includegraphics{TpromedioSinFuenteLibre.png}

\subsection{Frontera periodica}
La temperatura de una frontera es igual al borde de la placa opuesto a esta.

\includegraphics{TPlacaSinFuentePeriodicas0.png}
\includegraphics{TPlacaSinFuentePeriodicas100.png}
\includegraphics{TPlacaSinFuentePeriodicas2500.png}
\includegraphics{TpromedioSinFuentePeriodicas.png}

\section{Con fuente de temperatura}
En este caso hay una fuente de calor externa que mantiene la temperatura del rectangulo a 100°C a esta temperatura.
\subsection{Frontera fija}
La frontera de la placa se mantiene a 50°C independientemente de la temperatura de los punntos de la placa adyacente.

\includegraphics{TPlacaConFuenteFijas0.png}
\includegraphics{TPlacaConFuenteFijas100.png}
\includegraphics{TPlacaConFuenteFijas2500.png}
\includegraphics{TpromedioConFuenteFijas.png}

\subsection{Frontera libre}
La frontera de la placa tiene la misma temeratura que los nodos adyacentes a esta, por lo tanto la transferencia de calor en esta dirección es cero.

\includegraphics{TPlacaConFuenteLibre0.png}
\includegraphics{TPlacaConFuenteLibre100.png}
\includegraphics{TPlacaConFuenteLibre2500.png}
\includegraphics{TpromedioConFuenteLibre.png}

\subsection{Frontera periodica}
La temperatura de una frontera es igual al borde de la placa opuesto a esta.

\includegraphics{TPlacaConFuentePeriodicas0.png}
\includegraphics{TPlacaConFuentePeriodicas100.png}
\includegraphics{TPlacaConFuentePeriodicas2500.png}
\includegraphics{TpromedioConFuentePeriodicas.png}
\end{document}

